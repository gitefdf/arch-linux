\subsection{Plugin Spec}
\subsubsection{SPEC SOURCE}
\begin{tabularx}
\hline
Property&Type&Description\\\hline
[1]&string&Short plugin url. Will be expanded using config.git.url_format. Can
also be a url or dir.\\\hline
dir&string&A directory pointing to a local plugin\\\hline
url&string&A custom git url where the plugin is hosted\\\hline
name&string&A custom name for the plugin used for the local plugin directory
and as the display name\\\hline
dev&boolean&When true, a local plugin directory will be used instead. See
config.dev\\\hline
\end{tabularx}
A valid spec should define one of `[1]`, `dir` or `url`.
\subsubsection{SPEC LOADING}
\begin{tabular}
Property&Type&Description\\\hline
dependencies&LazySpec[]&一个插件名或插件规范的列表,它应该在插件加载时被加载.依赖项总是懒加载,除非另作说明.当是一个插件名时,确保已在别处定义过\\\hline
enabled&boolean or fun():boolen&当 enabled = false 或 function return false
时,这个插件不会被安装\\\hline
cond&boolean or fun(LazyPlugin):boolean&作用和 enabled 类似,区别是为 false 时,插件不会被卸载\\\hline
priority&number&仅作用于启动插件(lazy = false)时,以强制优先加载某插件。Default priority is
50.建议将此设置为高数字以用于配色方案。\\\hline
\end{tabular}

\textbf{dependencies}
\begin{listing}
return{
'plung-a'
dependencies = {
{'plung-b',
opts = {}
}
}
}
\end{listin}
\textbf{enabled}
\begin{}

\end{}






\subsection{SPEC SETUP}

Property&Type&Description\\\hline
init&fun(LazyPlugin)&init function 总是启动时执行. Mostly useful for setting vim.g. 通常是 vim 插件启动时读取的配置不等插件加载,立即执行\\\hline
opts&table or fun(LazyPlugin, opts:table)&opts 应当是是个表(会和父级spec合并),返回一个表(取代父级spec),或者改变一个表,这个表会被传递给 Plugin.config() function.设置这个值就包含 Plugin.config() function.\\\hline
config&fun(LazyPlugin, opts:table) or true&config会在插件加载时执行。如果设置了 opts or config = true ,会自动执行 require(MAIN).setup(opts),Lzay 会基于插件名自动推测出 plugin's MAIN module \\\hline
main&string&You can specify the main module to use for config() and opts(), in case it can not be determined automatically.See config()\\\hline
build& fun(LazyPlugin) or string or false or a list of build commands &build
会在插件安装或更新时执行,它的作用是:当插件被安装或更新时自动运行指定的构建命令或函数。See Building for more information.\\\hline
\subsection{SPEC LAZY LOADING}
\begin{}
Property&Type&Description\\\hline
lazy&boolean&lazy = true 时,插件仅在需要时被加载。\\\hline
event&string? or string[] or fun(self:LazyPlugin, event:string[]):string[] or {event:string[]|string, pattern?:string[]|string}&Lazy-load on event. Events can be specified as BufEnter or with a pattern like BufEnter *.lua\\\hline
cmd&string? or string[] or fun(self:LazyPlugin, cmd:string[]):string[]&Lazy-load on command\\\hline
ft&string? or string[] or fun(self:LazyPlugin, ft:string[]):string[]&Lazy-load on filetype\\\hline
keys&string? or string[] or LazyKeysSpec[] or fun(self:LazyPlugin, keys:string[]):(string | LazyKeysSpec)[]&Lazy-load on key mapping\\\hline
\end{}
Refer to the Lazy Loading <./lazy_loading.md> section for more information.


\subsection{SPEC VERSIONING}
\begin{}
&Property&Type&Description\\\hline
branch&string&Branch of the repository\\\hline
tag&string&Tag of the repository\\\hline
commit&string&Commit of the repository\\\hline
version&string? or false to override the default&Version to use from the repository. Full Semver ranges are supported\\\hline
pin&boolean&When true, this plugin will not be included in updates\\\hline
submodules&boolean&When false, git submodules will not be fetched. Defaults to true
\end{}

\subsection{SPEC ADVANCED}
Property&Type&Description
optional&boolean&&\\\hline
  specs      LazySpec  
  import     string? 
  ----------------------------------------------------------------------------------------

EXAMPLES


return {
-- the colorscheme should be available when starting Neovim
{
"folke/tokyonight.nvim",
lazy = false, -- make sure we load this during startup if it is your main colorscheme
priority = 1000, -- make sure to load this before all the other start plugins
config = function()
-- load the colorscheme here
vim.cmd([[colorscheme tokyonight]])
end,
},
-- I have a separate config.mappings file where I require which-key.
-- With lazy the plugin will be automatically loaded when it is required somewhere
{ "folke/which-key.nvim", lazy = true },

{
"nvim-neorg/neorg",
-- lazy-load on filetype
ft = "norg",
-- options for neorg. This will automatically call `require("neorg").setup(opts)`
opts = {
load = {
["core.defaults"] = {},
},
},
},

{
"dstein64/vim-startuptime",
-- lazy-load on a command
cmd = "StartupTime",
-- init is called during startup. Configuration for vim plugins typically should be set in an init function
init = function()
vim.g.startuptime_tries = 10
end,
},

{
"hrsh7th/nvim-cmp",
-- load cmp on InsertEnter
event = "InsertEnter",
-- these dependencies will only be loaded when cmp loads
-- dependencies are always lazy-loaded unless specified otherwise
dependencies = {
"hrsh7th/cmp-nvim-lsp",
"hrsh7th/cmp-buffer",
},
config = function()
-- ...
end,
},

-- if some code requires a module from an unloaded plugin, it will be automatically loaded.
-- So for api plugins like devicons, we can always set lazy=true
{ "nvim-tree/nvim-web-devicons", lazy = true },

-- you can use the VeryLazy event for things that can
-- load later and are not important for the initial UI
{ "stevearc/dressing.nvim", event = "VeryLazy" },

{
"Wansmer/treesj",
keys = {
{ "J", "<cmd>TSJToggle<cr>", desc = "Join Toggle" },
},
opts = { use_default_keymaps = false, max_join_length = 150 },
},

{
"monaqa/dial.nvim",
-- lazy-load on keys
-- mode is `n` by default. For more advanced options, check the section on key mappings
keys = { "<C-a>", { "<C-x>", mode = "n" } },
},

-- local plugins need to be explicitly configured with dir
{ dir = "~/projects/secret.nvim" },

-- you can use a custom url to fetch a plugin
{ url = "git@github.com:folke/noice.nvim.git" },

-- local plugins can also be configured with the dev option.
-- This will use {config.dev.path}/noice.nvim/ instead of fetching it from GitHub
-- With the dev option, you can easily switch between the local and installed version of a plugin
{ "folke/noice.nvim", dev = true },
}



\section{LAZY LOADING}
lazy.nvim 自动懒加载 lua 模块.这意味如果插件 A 是懒加载的,而插件 B 需要插件 A 的某个 Lua 模块,那么 A 会按需加载

此外,你也可以对 'events','commands','file types','key maps' 懒加载。

当出现以下情况之一时,插件将被懒加载。

- 插件仅作为 spec 中的依赖项

- It has an `event`, `cmd`, `ft` or `keys` key

- config.defaults.lazy == true

\subsection{LAZY KEY MAPPINGS}
The keys property can be a string or string[] for simple normal-mode mappings, or it can be a LazyKeysSpec table with the following key-value pairs:

- **[1]**: (`string`) lhs **(required)**
- **[2]**: (`string|fun()`) rhs **(optional)**
- **mode**: (`string|string[]`) mode **(optional, defaults to "n")**
- **ft**: (`string|string[]`) `filetype` for buffer-local keymaps **(optional)**
- any other option valid for `vim.keymap.set`

Key mappings will load the plugin the first time they get executed.

When `[2]` is `nil`, then the real mapping has to be created by the `config()`
function.


-- Example for neo-tree.nvim
{
"nvim-neo-tree/neo-tree.nvim",
keys = {
{ "<leader>ft", "<cmd>Neotree toggle<cr>", desc = "NeoTree" },
},
opts = {},
}



\section{VERSIONING}

If you want to install a specific revision of a plugin, you can use `commit`,`tag`, `branch`, `version`.

The `version` property supports Semver <https://semver.org/> ranges.

EXAMPLES 
\begin{ve}
- `*`: latest stable version (this excludes pre-release versions)
- `1.2.x`: any version that starts with `1.2`, such as `1.2.0`, `1.2.3`, etc.
- `^1.2.3`: any version that is compatible with `1.2.3`, such as `1.3.0`, `1.4.5`, etc., but not `2.0.0`.
- `~1.2.3`: any version that is compatible with `1.2.3`, such as `1.2.4`, `1.2.5`, but not `1.3.0`.
- `>1.2.3`: any version that is greater than `1.2.3`, such as `1.3.0`, `1.4.5`, etc.
- `>=1.2.3`: any version that is greater than or equal to `1.2.3`, such as `1.2.3`, `1.3.0`, `1.4.5`, etc.
- `<1.2.3`: any version that is less than `1.2.3`, such as `1.1.0`, `1.0.5`, etc.
- `<=1.2.3`: any version that is less than or equal to `1.2.3`, such as `1.2.3`, `1.1.0`, `1.0.5`, etc
\end{ve}


\section{Configuration}
lazy.nvim comes with the following defaults:

{
root = vim.fn.stdpath("data") .. "/lazy", -- directory where plugins will be installed
defaults = {
-- Set this to `true` to have all your plugins lazy-loaded by default.
-- Only do this if you know what you are doing, as it can lead to unexpected behavior.
lazy = false, -- should plugins be lazy-loaded?
-- It's recommended to leave version=false for now, since a lot the plugin that support versioning,
-- have outdated releases, which may break your Neovim install.
version = nil, -- always use the latest git commit
-- version = "*", -- try installing the latest stable version for plugins that support semver
-- default `cond` you can use to globally disable a lot of plugins
-- when running inside vscode for example
cond = nil, ---@type boolean|fun(self:LazyPlugin):boolean|nil
},
-- leave nil when passing the spec as the first argument to setup()
spec = nil, ---@type LazySpec
local_spec = true, -- load project specific .lazy.lua spec files. They will be added at the end of the spec.
lockfile = vim.fn.stdpath("config") .. "/lazy-lock.json", -- lockfile generated after running update.
---@type number? limit the maximum amount of concurrent tasks
concurrency = jit.os:find("Windows") and (vim.uv.available_parallelism() * 2) or nil,
git = {
-- defaults for the `Lazy log` command
-- log = { "--since=3 days ago" }, -- show commits from the last 3 days
log = { "-8" }, -- show the last 8 commits
timeout = 120, -- kill processes that take more than 2 minutes
url_format = "https://github.com/%s.git",
-- lazy.nvim requires git >=2.19.0. If you really want to use lazy with an older version,
-- then set the below to false. This should work, but is NOT supported and will
-- increase downloads a lot.
filter = true,
-- rate of network related git operations (clone, fetch, checkout)
throttle = {
enabled = false, -- not enabled by default
-- max 2 ops every 5 seconds
rate = 2,
duration = 5 * 1000, -- in ms
},
-- Time in seconds to wait before running fetch again for a plugin.
-- Repeated update/check operations will not run again until this
-- cooldown period has passed.
cooldown = 0,
},
pkg = {
enabled = true,
cache = vim.fn.stdpath("state") .. "/lazy/pkg-cache.lua",
-- the first package source that is found for a plugin will be used.
sources = {
"lazy",
"rockspec", -- will only be used when rocks.enabled is true
"packspec",
},
},
rocks = {
enabled = true,
root = vim.fn.stdpath("data") .. "/lazy-rocks",
server = "https://nvim-neorocks.github.io/rocks-binaries/",
-- use hererocks to install luarocks?
-- set to `nil` to use hererocks when luarocks is not found
-- set to `true` to always use hererocks
-- set to `false` to always use luarocks
hererocks = nil,
},
dev = {
-- Directory where you store your local plugin projects. If a function is used,
-- the plugin directory (e.g. `~/projects/plugin-name`) must be returned.
---@type string | fun(plugin: LazyPlugin): string
path = "~/projects",
---@type string[] plugins that match these patterns will use your local versions instead of being fetched from GitHub
patterns = {}, -- For example {"folke"}
fallback = false, -- Fallback to git when local plugin doesn't exist
},
install = {
-- install missing plugins on startup. This doesn't increase startup time.
missing = true,
-- try to load one of these colorschemes when starting an installation during startup
colorscheme = { "habamax" },
},
ui = {
-- a number <1 is a percentage., >1 is a fixed size
size = { width = 0.8, height = 0.8 },
wrap = true, -- wrap the lines in the ui
-- The border to use for the UI window. Accepts same border values as |nvim_open_win()|.
border = "none",
-- The backdrop opacity. 0 is fully opaque, 100 is fully transparent.
backdrop = 60,
title = nil, ---@type string only works when border is not "none"
title_pos = "center", ---@type "center" | "left" | "right"
-- Show pills on top of the Lazy window
pills = true, ---@type boolean
icons = {
cmd = " ",
config = "",
debug = "● ",
event = " ",
favorite = " ",
ft = " ",
init = " ",
import = " ",
keys = " ",
lazy = "󰒲 ",
loaded = "●",
not_loaded = "○",
plugin = " ",
runtime = " ",
require = "󰢱 ",
source = " ",
start = " ",
task = "✔ ",
list = {
"●",
"➜",
"★",
"‒",
},
},
-- leave nil, to automatically select a browser depending on your OS.
-- If you want to use a specific browser, you can define it here
browser = nil, ---@type string?
throttle = 1000 / 30, -- how frequently should the ui process render events
custom_keys = {
-- You can define custom key maps here. If present, the description will
-- be shown in the help menu.
-- To disable one of the defaults, set it to false.

["<localleader>l"] = {
function(plugin)
require("lazy.util").float_term({ "lazygit", "log" }, {
cwd = plugin.dir,
})
end,
desc = "Open lazygit log",
},

["<localleader>i"] = {
function(plugin)
Util.notify(vim.inspect(plugin), {
title = "Inspect " .. plugin.name,
lang = "lua",
})
end,
desc = "Inspect Plugin",
},

["<localleader>t"] = {
function(plugin)
require("lazy.util").float_term(nil, {
cwd = plugin.dir,
})
end,
desc = "Open terminal in plugin dir",
},
},
},
-- Output options for headless mode
headless = {
-- show the output from process commands like git
process = true,
-- show log messages
log = true,
-- show task start/end
task = true,
-- use ansi colors
colors = true,
},
diff = {
-- diff command <d> can be one of:
-- * browser: opens the github compare view. Note that this is always mapped to <K> as well,
--   so you can have a different command for diff <d>
-- * git: will run git diff and open a buffer with filetype git
-- * terminal_git: will open a pseudo terminal with git diff
-- * diffview.nvim: will open Diffview to show the diff
cmd = "git",
},
checker = {
-- automatically check for plugin updates
enabled = false,
concurrency = nil, ---@type number? set to 1 to check for updates very slowly
notify = true, -- get a notification when new updates are found
frequency = 3600, -- check for updates every hour
check_pinned = false, -- check for pinned packages that can't be updated
},
change_detection = {
-- automatically check for config file changes and reload the ui
enabled = true,
notify = true, -- get a notification when changes are found
},
performance = {
cache = {
enabled = true,
},
reset_packpath = true, -- reset the package path to improve startup time
rtp = {
reset = true, -- reset the runtime path to $VIMRUNTIME and your config directory
---@type string[]
paths = {}, -- add any custom paths here that you want to includes in the rtp
---@type string[] list any plugins you want to disable here
disabled_plugins = {
-- "gzip",
-- "matchit",
-- "matchparen",
-- "netrwPlugin",
-- "tarPlugin",
-- "tohtml",
-- "tutor",
-- "zipPlugin",
},
},
},
-- lazy can generate helptags from the headings in markdown readme files,
-- so :help works even for plugins that don't have vim docs.
-- when the readme opens with :help it will be correctly displayed as markdown
readme = {
enabled = true,
root = vim.fn.stdpath("state") .. "/lazy/readme",
files = { "README.md", "lua/**/README.md" },
-- only generate markdown helptags for plugins that don't have docs
skip_if_doc_exists = true,
},
state = vim.fn.stdpath("state") .. "/lazy/state.json", -- state info for checker and other things
-- Enable profiling of lazy.nvim. This will add some overhead,
-- so only enable this when you are debugging lazy.nvim
profiling = {
-- Enables extra stats on the debug tab related to the loader cache.
-- Additionally gathers stats about all package.loaders
loader = false,
-- Track each new require in the Lazy profiling tab
require = false,
},
}

If you don’t want to use a Nerd Font, you can replace the icons with Unicode symbols. ~

lua
{
ui = {
icons = {
cmd = "⌘",
config = "🛠",
event = "📅",
ft = "📂",
init = "⚙",
keys = "🗝",
plugin = "🔌",
runtime = "💻",
require = "🌙",
source = "📄",
start = "🚀",
task = "📌",
lazy = "💤 ",
},
},
}


\section{Usage}

▶️ STARTUP SEQUENCE         *lazy.nvim-🚀-usage-▶️-startup-sequence*


lazy.nvim不使用 Neovim 软件包,甚至完全禁用插件加载 (`vim.go.loadplugins = false`).它接管整个启动过程,以获得更大的灵活性和更好的性能.
In practice this means that step 10 of |Neovim Initialization| is done by Lazy:

1. All the plugins’ `init()` functions are executed
2. All plugins with `lazy=false` are loaded. This includes sourcing `/plugin` and `/ftdetect` files. (`/after` will not be sourced yet)
3. All files from `/plugin` and `/ftdetect` directories in your rtp are sourced (excluding `/after`)
4. All `/after/plugin` files are sourced (this includes `/after` from plugins)

Files from runtime directories are always sourced in alphabetical order.


🚀 COMMANDS                             *lazy.nvim-🚀-usage-🚀-commands*

Plugins are managed with the `:Lazy` command. Open the help with `<?>` to see
all the key mappings.

You can press `<CR>` on a plugin to show its details. Most properties can be
hovered with `<K>` to open links, help files, readmes, git commits and git
issues.

Lazy can automatically check for updates in the background. This feature can be
enabled with `config.checker.enabled = true`.

Any operation can be started from the UI, with a sub command or an API
function:

  ----------------------------------------------------------------------------------
  Command                   Lua                              Description
  ------------------------- -------------------------------- -----------------------
  :Lazy build {plugins}     require("lazy").build(opts)      Rebuild a plugin

  :Lazy check [plugins]     require("lazy").check(opts?)     Check for updates and
                                                             show the log (git
                                                             fetch)

  :Lazy clean [plugins]     require("lazy").clean(opts?)     Clean plugins that are
                                                             no longer needed

  :Lazy clear               require("lazy").clear()          Clear finished tasks

  :Lazy debug               require("lazy").debug()          Show debug information

  :Lazy health              require("lazy").health()         Run :checkhealth lazy

  :Lazy help                require("lazy").help()           Toggle this help page

  :Lazy home                require("lazy").home()           Go back to plugin list

  :Lazy install [plugins]   require("lazy").install(opts?)   Install missing plugins

  :Lazy load {plugins}      require("lazy").load(opts)       Load a plugin that has
                                                             not been loaded yet.
                                                             Similar to :packadd.
                                                             Like
                                                             :Lazy load foo.nvim.
                                                             Use :Lazy! load to skip
                                                             cond checks.

  :Lazy log [plugins]       require("lazy").log(opts?)       Show recent updates

  :Lazy profile             require("lazy").profile()        Show detailed profiling

  :Lazy reload {plugins}    require("lazy").reload(opts)     Reload a plugin
                                                             (experimental!!)

  :Lazy restore [plugins]   require("lazy").restore(opts?)   Updates all plugins to
                                                             the state in the
                                                             lockfile. For a single
                                                             plugin: restore it to
                                                             the state in the
                                                             lockfile or to a given
                                                             commit under the cursor

  :Lazy sync [plugins]      require("lazy").sync(opts?)      Run install, clean and
                                                             update

  :Lazy update [plugins]    require("lazy").update(opts?)    Update plugins. This
                                                             will also update the
                                                             lockfile
  ----------------------------------------------------------------------------------
Any command can have a **bang** to make the command wait till it finished. For
example, if you want to sync lazy from the cmdline, you can use:

>shell
    nvim --headless "+Lazy! sync" +qa
<

`opts` is a table with the following key-values:

- **wait**: when true, then the call will wait till the operation completed
- **show**: when false, the UI will not be shown
- **plugins**: a list of plugin names to run the operation on
- **concurrency**: limit the `number` of concurrently running tasks

Stats API (`require("lazy").stats()`):

>lua
{
-- startuptime in milliseconds till UIEnter
startuptime = 0,
-- when true, startuptime is the accurate cputime for the Neovim process. (Linux & macOS)
-- this is more accurate than `nvim --startuptime`, and as such will be slightly higher
-- when false, startuptime is calculated based on a delta with a timestamp when lazy started.
real_cputime = false,
count = 0, -- total number of plugins
loaded = 0, -- number of loaded plugins
---@type table<string, number>
times = {},
}
<

**lazy.nvim** provides a statusline component that you can use to show the
number of pending updates. Make sure to enable `config.checker.enabled = true`
to make this work.

Example of configuring lualine.nvim ~

>lua
require("lualine").setup({
sections = {
lualine_x = {
{
require("lazy.status").updates,
cond = require("lazy.status").has_updates,
color = { fg = "#ff9e64" },
},
},
},
})
<


📆 USER EVENTS                       *lazy.nvim-🚀-usage-📆-user-events*

The following user events will be triggered:

- **LazyDone**: when lazy has finished starting up and loaded your config
- **LazySync**: after running sync
- **LazyInstall**: after an install
- **LazyUpdate**: after an update
- **LazyClean**: after a clean
- **LazyCheck**: after checking for updates
- **LazyLog**: after running log
- **LazyLoad**: after loading a plugin. The `data` attribute will contain the plugin name.
- **LazySyncPre**: before running sync
- **LazyInstallPre**: before an install
- **LazyUpdatePre**: before an update
- **LazyCleanPre**: before a clean
- **LazyCheckPre**: before checking for updates
- **LazyLogPre**: before running log
- **LazyReload**: triggered by change detection after reloading plugin specs
- **VeryLazy**: triggered after `LazyDone` and processing `VimEnter` auto commands
- **LazyVimStarted**: triggered after `UIEnter` when `require("lazy").stats().startuptime` has been calculated.
    Useful to update the startuptime on your dashboard.


❌ UNINSTALLING                       *lazy.nvim-🚀-usage-❌-uninstalling*

To uninstall **lazy.nvim**, you need to remove the following files and
directories:

- **data**: `~/.local/share/nvim/lazy`
- **state**: `~/.local/state/nvim/lazy`
- **lockfile**: `~/.config/nvim/lazy-lock.json`


  Paths can differ if you changed `XDG` environment variables.

🔒 LOCKFILE                             *lazy.nvim-🚀-usage-🔒-lockfile*

After every **update**, the local lockfile (`lazy-lock.json`) is updated with
the installed revisions. It is recommended to have this file under version
control.

If you use your Neovim config on multiple machines, using the lockfile, you can
ensure that the same version of every plugin is installed.

If you are on another machine, you can do `:Lazy restore`, to update all your
plugins to the version from the lockfile.


📦 MIGRATION GUIDE               *lazy.nvim-🚀-usage-📦-migration-guide*


PACKER.NVIM ~

- `setup` ➡️ `init`
- `requires` ➡️ `dependencies`
- `as` ➡️ `name`
- `opt` ➡️ `lazy`
- `run` ➡️ `build`
- `lock` ➡️ `pin`
- `disable=true` ➡️ `enabled = false`
- `tag='*'` ➡️ `version="*"`
- `after` is **not needed** for most use-cases. Use `dependencies` otherwise.
- `wants` is **not needed** for most use-cases. Use `dependencies` otherwise.
- `config` don’t support string type, use `fun(LazyPlugin)` instead.
- `module` is auto-loaded. No need to specify
- `keys` spec is |lazy.nvim-different|
- `rtp` can be accomplished with:

>lua
    config = function(plugin)
        vim.opt.rtp:append(plugin.dir .. "/custom-rtp")
    end
<

With packer `wants`, `requires` and `after` can be used to manage dependencies.
With lazy, this isn’t needed for most of the Lua dependencies. They can be
installed just like normal plugins (even with `lazy=true`) and will be loaded
when other plugins need them. The `dependencies` key can be used to group those
required plugins with the one that requires them. The plugins which are added
as `dependencies` will always be lazy-loaded and loaded when the plugin is
loaded.


PAQ-NVIM ~

- `as` ➡️ `name`
- `opt` ➡️ `lazy`
- `run` ➡️ `build`


⚡ PROFILING & DEBUG             *lazy.nvim-🚀-usage-⚡-profiling-&-debug*

Great care has been taken to make the startup code (`lazy.core`) as efficient
as possible. During startup, all Lua files used before `VimEnter` or
`BufReadPre` are byte-compiled and cached, similar to what impatient.nvim
<https://github.com/lewis6991/impatient.nvim> does.

My config for example loads in about `11ms` with `93` plugins. I do a lot of
lazy-loading though :)

**lazy.nvim** comes with an advanced profiler `:Lazy profile` to help you
improve performance. The profiling view shows you why and how long it took to
load your plugins.


🐛 DEBUG ~

See an overview of active lazy-loading handlers and what’s in the module
cache.


📂 STRUCTURING YOUR PLUGINS*lazy.nvim-🚀-usage-📂-structuring-your-plugins*

Some users may want to split their plugin specs in multiple files. Instead of
passing a spec table to `setup()`, you can use a Lua module. The specs from the
**module** and any top-level **sub-modules** will be merged together in the
final spec, so it is not needed to add `require` calls in your main plugin file
to the other files.

The benefits of using this approach:

- Simple to **add** new plugin specs. Just create a new file in your plugins module.
- Allows for **caching** of all your plugin specs. This becomes important if you have a lot of smaller plugin specs.
- Spec changes will automatically be **reloaded** when they’re updated, so the `:Lazy` UI is always up to date.

Example:

- `~/.config/nvim/init.lua`

>lua
    require("lazy").setup("plugins")
<

- `~/.config/nvim/lua/plugins.lua` or `~/.config/nvim/lua/plugins/init.lua` **(this file is optional)**

>lua
return {
"folke/neodev.nvim",
"folke/which-key.nvim",
{ "folke/neoconf.nvim", cmd = "Neoconf" },
}
<

- Any lua file in `~/.config/nvim/lua/plugins/*.lua` will be automatically merged in the main plugin spec

For a real-life example, you can check LazyVim
<https://github.com/LazyVim/LazyVim> and more specifically:

- lazyvim.plugins <https://github.com/LazyVim/LazyVim/tree/main/lua/lazyvim/plugins> contains all the plugin specs that will be loaded


↩️ IMPORTING SPECS, CONFIG & OPTS

As part of a spec, you can add `import` statements to import additional plugin
modules. Both of the `setup()` calls are equivalent:

>lua
require("lazy").setup("plugins")

-- Same as:
require("lazy").setup({{import = "plugins"}})
<

To import multiple modules from a plugin, add additional specs for each import.
For example, to import LazyVim core plugins and an optional plugin:

>lua
require("lazy").setup({
spec = {
{ "LazyVim/LazyVim", import = "lazyvim.plugins" },
{ import = "lazyvim.plugins.extras.coding.copilot" },
}
})
<

When you import specs, you can override them by simply adding a spec for the
same plugin to your local specs, adding any keys you want to override / merge.

`opts`, `dependencies`, `cmd`, `event`, `ft` and `keys` are always merged with
the parent spec. Any other property will override the property from the parent
spec.


==============================================================================
8. 🔥 Developers                                 *lazy.nvim-🔥-developers*

To make it easier for users to install your plugin, you can include a package
spec </packages> in your repo.


BEST PRACTICES                      *lazy.nvim-🔥-developers-best-practices*

- If your plugin needs `setup()`, then create a simple `lazy.lua` file like this:
    >lua
          return { "me/my-plugin", opts = {} }
    <
- Plugins that are pure lua libraries should be lazy-loaded with `lazy = true`.
    >lua
        { "nvim-lua/plenary.nvim", lazy = true }
    <
- Always use `opts` instead of `config` when possible. `config` is almost never
    needed.
- Only use `dependencies` if a plugin needs the dep to be installed **AND**
    loaded. Lua plugins/libraries are automatically loaded when they are
    `require()`d, so they don’t need to be in `dependencies`.
- Inside a `build` function or `*.lua` build file, use
    `coroutine.yield(msg:string|LazyMsg)` to show progress.
- Don’t change the `cwd` in your build function, since builds run in parallel
    and changing the `cwd` will affect other builds.


BUILDING                                  *lazy.nvim-🔥-developers-building*

The spec **build** property can be one of the following:

- `fun(plugin: LazyPlugin)`: a function that builds the plugin.
- `*.lua`: a Lua file that builds the plugin (like `build.lua`)
- `":Command"`: a Neovim command
- `"rockspec"`: this will run `luarocks make` in the plugin’s directory
    This is automatically set by the `rockspec` package </packages> source.
- any other **string** will be run as a shell command
- a `list` of any of the above to run multiple build steps
- if no `build` is specified, but a `build.lua` file exists, that will be used instead.

Build functions and `*.lua` files run asynchronously in a coroutine. Use
`coroutine.yield(msg:string|LazyMsg)` to show progress.

Yielding will also schedule the next `coroutine.resume()` to run in the next
tick, so you can do long-running tasks without blocking Neovim.

>lua
    ---@class LazyMsg
    ---@field msg string
    ---@field level? number vim.log.levels.XXX
<

Use `vim.log.levels.TRACE` to only show the message as a **status** message for
the task.



MINIT (MINIMAL INIT)          *lazy.nvim-🔥-developers-minit-(minimal-init)*

**lazy.nvim** comes with some built-in functionality to help you create a
minimal init for your plugin.

I mainly use this for testing and for users to create a `repro.lua`.

When running in **headless** mode, **lazy.nvim** will log any messages to the
terminal. See `opts.headless` for more info.

**minit** will install/load all your specs and will always run an update as
well.


BOOTSTRAP ~

>lua
    -- setting this env will override all XDG paths
    vim.env.LAZY_STDPATH = ".tests"
    -- this will install lazy in your stdpath
    load(vim.fn.system("curl -s https://raw.githubusercontent.com/folke/lazy.nvim/main/bootstrap.lua"))()
<


TESTING WITH BUSTED ~

This will add `"lunarmodules/busted"`, configure `hererocks` and run `busted`.

Below is an example of how I use **minit** to run tests with busted
<https://olivinelabs.com/busted/> in **LazyVim**.

>lua
#!/usr/bin/env -S nvim -l

vim.env.LAZY_STDPATH = ".tests"
load(vim.fn.system("curl -s https://raw.githubusercontent.com/folke/lazy.nvim/main/bootstrap.lua"))()

-- Setup lazy.nvim
require("lazy.minit").busted({
spec = {
"LazyVim/starter",
"williamboman/mason-lspconfig.nvim",
"williamboman/mason.nvim",
"nvim-treesitter/nvim-treesitter",
},
})
<

To use this, you can run:

>sh
    nvim -l ./tests/busted.lua tests
<

If you want to inspect the test environment, run:

>sh
    nvim -u ./tests/busted.lua
<


REPRO.LUA ~

>lua
vim.env.LAZY_STDPATH = ".repro"
load(vim.fn.system("curl -s https://raw.githubusercontent.com/folke/lazy.nvim/main/bootstrap.lua"))()

require("lazy.minit").repro({
spec = {
"stevearc/conform.nvim",
"nvim-neotest/nvim-nio",
},
})

-- do anything else you need to do to reproduce the issue
<

Then run it with:

>sh
    nvim -u repro.lua
<

==============================================================================
9. Links                                                     *lazy.nvim-links*

1. *image*: https://user-images.githubusercontent.com/292349/208301737-68fb279c-ba70-43ef-a369-8c3e8367d6b1.png
2. *image*: https://user-images.githubusercontent.com/292349/208301766-5c400561-83c3-4811-9667-1ec4bb3c43b8.png
3. *image*: https://user-images.githubusercontent.com/292349/208301790-7eedbfa5-d202-4e70-852e-de68aa47233b.png

Generated by panvimdoc <https://github.com/kdheepak/panvimdoc>

vim:tw=78:ts=8:noet:ft=help:norl:
